\documentclass{article}
\usepackage[left=3cm,top=3cm,right=4cm,nohead,bottom=3cm]{geometry}
%THIS FILE IS AUTOMATICALLY GENERATED. DO NOT EDIT.
\usepackage{amsmath}
\usepackage{amssymb}
\usepackage{proof}
\usepackage{hyperref}
\newtheorem{lemma}{Lemma}
\newtheorem{corr}{Corollary}
\newenvironment{hint}{\noindent{\bf (Hint)}}{}
\newtheorem{thm}{Theorem}
\newtheorem{definition}{Definition}
\newtheorem{task}{Task}
\newtheorem{bonus}{Bonus Task}

\newif\ifsol

\newcommand{\makehwtitle}[4]
{{\centering\Large\textbf{Homework #1: #2}
    \normalsize\\
15-814: Types and Programming Languages\\
Fall 2015\\
TA: Evan Cavallo (ecavallo@cs.cmu.edu)\\
\ifsol \solcolor \textbf{SOLUTIONS} \normalcolor \\ \fi \vspace{0.5cm}%
Out: #3\\
Due: #4\\
}}

\newenvironment{proof}{\trivlist \item[\hskip \labelsep{\bf
Proof:}]}{\hfill$\Box$ \endtrivlist}

\newenvironment{sol}{
\trivlist \item[\hskip \labelsep{\bf
Solution:}]}{\endtrivlist}

\newcommand{\solution}[1]{\ifsol \begin{sol} #1 \end{sol} \fi}

\newcommand{\taskpts}[1]{[#1]}

\newenvironment{includeintemplate}{}{}
\newenvironment{noinclude}{}{}
\newcommand{\rulename}[1]{\text{\textsc{(#1)}}}
\newcommand{\einfer}[3]
        {\begin{equation}\label{eq:#1}\infer[]{#2}{#3}\end{equation}}

\allowdisplaybreaks       %mildly permissible to break up displayed equations
% Problem 1

\newcommand{\defeq}{\overset{\mathsf{def}}{=}}

%%Expression language
%Sorts
\newcommand{\sort}{\mathsf}
\newcommand{\exps}{\sort{exp}}
\newcommand{\funs}{\sort{fun}}
\newcommand{\progs}{\sort{prog}}

%Syntax
\newcommand{\elangname}{\textbf{E}}
\newcommand{\opf}[1]{\mathtt{#1}}
\newcommand{\numt}{\opf{int}}
\newcommand{\boolt}{\opf{bool}}
\newcommand{\Arrt}[2]{#1 \Rightarrow #2}
\newcommand{\num}[1]{\opf{num}[#1]}
\newcommand{\etrue}{\opf{true}}
\newcommand{\efalse}{\opf{false}}
\newcommand{\eplus}[2]{\opf{plus}(#1; #2)}
\newcommand{\etimes}[2]{\opf{times}(#1; #2)}
\newcommand{\eleq}[2]{\opf{leq}(#1; #2)}
\newcommand{\eifb}[3]{\opf{if}(#1; #2; #3)}
\newcommand{\cifb}[3]{\opf{if}~#1~\opf{then}~#2~\opf{else}~#3}
\newcommand{\efun}[1]{\opf{fn}(#1)}
\newcommand{\eappe}[2]{{^*}#1(#2)}
\newcommand{\metacat}{\text{\textasciicircum}}

\newcommand{\pfun}[4]{\opf{fun}~#1~(#2 : #3)~\{ #4 \};}
\newcommand{\pres}[1]{\opf{result}~#1}
\newcommand{\pstruct}[2]{\opf{type}~#1 = #2;}

%Judgments
\newcommand{\G}{\Gamma}
\newcommand{\D}{\Delta}
\newcommand{\entails}{\vdash}
\newcommand{\etype}[2]{#1 :_{\mathsf{e}} #2}
\newcommand{\ftype}[3]{#1 :_{\mathsf{f}} \Arrt{#2}{#3}}
\newcommand{\ptype}[2]{#1 :_{\mathsf{p}} #2}
\newcommand{\fctx}[2]{#1 ::_{\mathsf{f}} #2}

\newcommand{\stepsto}{\mapsto}
\newcommand{\val}{\mathsf{val}}
\newcommand{\esteps}[2]{#1 \stepsto_{\mathsf{e}} #2}
\newcommand{\psteps}[2]{#1 \stepsto_{\mathsf{p}} #2}
\newcommand{\eval}[1]{#1\; \val_{\mathsf{e}}}
\newcommand{\pval}[1]{#1\; \val_{\mathsf{p}}}
\newcommand{\fundef}[3]{#1 \defeq #2.#3}

\newcommand{\varname}{\textit}

%Types
\newcommand{\unitt}{\mathtt{unit}}
\newcommand{\sumt}[2]{#1 + #2}
\newcommand{\prodt}[2]{#1 \times #2}

%Expressions
\newcommand{\inl}[1]{\mathtt{in}[l](#1)}
\newcommand{\inr}[1]{\mathtt{in}[r](#1)}
\newcommand{\ecase}[5]{\mathtt{case}\; #1\; \{#2. #3; #4. #5\}}
\newcommand{\eprod}[2]{\langle #1, #2 \rangle}
\newcommand{\triv}{\langle \rangle}
\newcommand{\projl}[1]{\mathtt{fst}\; #1}
\newcommand{\projr}[1]{\mathtt{snd}\; #1}


\title{15-814 Homework hw1}
\author{YOUR NAME HERE}

\begin{document}
\maketitle

\section{Arithmetic}
\begin{task}
Prove the following \emph{inversion lemma}:
\begin{quote}
  (If Inversion) If $\G \entails \etype{\eifb{e_1}{e_2}{e_3}}{\tau}$,
  then $\G \entails \etype{e_1}{\boolt}$, $\G \entails
  \etype{e_2}{\tau}$, and $\G \entails \etype{e_3}{\tau}$.
\end{quote}
This seems immediate, but really follows from the induction principle
for the typing judgment. (Tip: Prove that for all $\G$, $e$, $\tau$
such that $\G \entails \etype{e}{\tau}$, if $e = \eifb{e_1}{e_2}{e_3}$
for some $e_1$, $e_2$, $e_3$, then $\G \entails \etype{e_1}{\boolt}$,
$\G \entails \etype{e_2}{\tau}$, and $\G \entails \etype{e_3}{\tau}$.)
\end{task}
\begin{sol}
\end{sol}

\begin{task}
Prove unicity of typing for this language.
\begin{quote}
  \emph{(Unicity of Typing)} For any $\G$, $e$, $\tau$, $\tau'$ such
  that $\G \entails \etype{e}{\tau}$ and $\G \entails
  \etype{e}{\tau'}$, we have $\tau = \tau'$. You may assume that any
  variable appears at most once in a given context.
\end{quote}
\end{task}
\begin{sol}
\end{sol}

\section{Functions}
\subsection{Function Blocks} \label{sec:function-blocks}
\begin{task} \label{task:syntax}
  Give a definition in abstract syntax for the language constructs
  described above. Include the arities (with sorts) of the operators,
  per Section 1.2 of PFPL.
\end{task}
\begin{sol}
\end{sol}

\begin{task} \label{task:scope}
  We want to respect the \emph{identification convention}:
  \begin{quote}
    Abstract binding trees are always identified up to $\alpha$-equivalence.
  \end{quote}
  With this in mind, which of the two strategies is correct? Explain.
\end{task}
\begin{sol}
\end{sol}

\begin{task} \label{task:typing} Starting with the rules in Appendix
  \ref{app:exp}, complete the definition of the typing judgment $\G
  \entails \etype{e}{\tau}$, and define the judgement
  $\G \entails \ptype{p}{\tau}$.
\end{task}
\begin{sol}
\end{sol}

\begin{task} \label{task:sos}
  Give a structural operational semantics for this language by
  completing the definition of $\D \entails \esteps{e}{e'}$ and
  defining a judgment $\D \entails \psteps{p}{p'}$. Assume that the
  value judgment $\D \entails \pval{p}$ is defined by the single rule
  \[
  \infer[\rulename{Res-V}]
    {\D \entails \pval{(\pres{e})}}
    {\D \entails \eval{e}}
  \]

  In order to track the environment of function definitions, we will
  add a new kind of assumption to our evaluation context:
  \[
  \begin{array}{lcl}
    \D & ::= & \cdots \mid \fundef{u}{x}{e}
  \end{array}
  \]
  The assumption $\fundef{u}{x}{e}$ expresses that the function
  variable $u$ is currently in scope and that it takes an argument $x$
  and computes the expression $e$.

  Your definition should of the semantics should satisfy the
  properties of \emph{progress} and \emph{preservation}.

  \begin{quote}
    \emph{(Progress)} If $\entails \ptype{p}{\tau}$, then either
    $\entails \pval{p}$ or there exists $p'$ such that
    $\entails \psteps{p}{p'}$. \\
    \emph{(Preservation)} If $\entails \ptype{p}{\tau}$ and $\entails
    \psteps{p}{p'}$, then $\entails \ptype{p'}{\tau}$.
  \end{quote}
\end{task}
\begin{sol}
\end{sol}

\begin{task}
  Prove progress for the rules you have specified. You will want to
  state and prove an analogous theorem for expressions first. For
  programs, it may be helpful to prove the following more general
  theorem:
  \begin{quote}
    If $\fctx{\D}{\G}$ and $\G \entails \ptype{p}{\tau}$, then either $\D
    \entails \pval{p}$ or there exists $p'$ such that
    $\D \entails \psteps{p}{p'}$. \\
  \end{quote}
  Here $\fctx{\D}{\G}$ is a judgment on contexts defined by the
  following rules:
  \[
  \infer[\rulename{Env-Nil}]
    {\fctx{\cdot}{\cdot}}
    {}
  \qquad
  \infer[\rulename{Env-Cons}]
    {\fctx{(\D, \fundef{f}{x}{e})}{(\G, \ftype{f}{\tau_1}{\tau_2})}}
    {\fctx{\D}{\G}}
  \]
  This judgment expresses that the environment $\D$ and type context
  $\G$ reference the same function variables, i.e. that they agree on
  what is in scope. This condition is enough to guarantee a canonical
  forms lemma, which you will need for your proof:
  \begin{quote}
    \emph{(Canonical Forms Lemma)} Let contexts $\fctx{\D}{\G}$ and an
    expression $e$ be given. Assume that $\D \entails \eval{e}$. If
    $\G \entails \etype{e}{\numt}$, then $e = \num{n}$ for some $n \in
    \mathbb{Z}$. If $\G \entails \etype{e}{\boolt}$, then either $e =
    \etrue$ or $e = \efalse$.
  \end{quote}
  This lemma enumerates the forms that well-typed values can take (the
  eponymous \emph{canonical forms}). You are not required to prove
  this lemma.
\end{task}
\begin{sol}
\end{sol}

\begin{task} \label{task:recursive}
  How does this extension change your answers to Tasks
  \ref{task:syntax}, \ref{task:typing}, and \ref{task:sos}? Give the
  syntax and rules where different. (You should also make a change to
  the form of the assumption $\fundef{u}{x}{e}$.)
\end{task}
\begin{sol}
\end{sol}

\subsection{Functions as Expressions}
\begin{task} \label{task:fix}
  Which property (progress or preservation) fails? Why? Describe two
  solutions to the problem, one by changing the rules for the
  judgments $\D \vdash \pval{p}$ and $\D \vdash \psteps{p}{p'}$ and
  one by changing the rules for the judgment $\D \vdash \ptype{p}{\tau}$.
\end{task}
\begin{sol}
\end{sol}

\begin{task} \label{task:lambda} Return to the core expression
  language. We add a new, distinct function type which contains
  recursive function expressions.
  \[
    \tau ::= \cdots \mid \tau \to \tau
  \]
  Define abstract syntax for operators for constructing and applying
  recursive function terms. (In other words, define a version of
  $\lambda$-abstraction and application which allows for
  self-reference.) Give statics and dynamics rules for these
  operators, assuming that we have only the expression language and no
  outer layer of function definitions.
\end{task}
\begin{sol}
\end{sol}

\section{Structs}
\begin{task}
  Give a type declaration in this language which corresponds to the C type \verb|tree| defined above. This definition should match the C struct's behavior exactly.
\end{task}
\begin{sol}
\end{sol}

\begin{task}
  The C struct definition of \verb|tree| assumes that the programmer
  follows the conventions described above, and it is possible to
  construct a \verb|tree| which is well-typed but corresponds to no
  legal tree. Give an alternate definition of \verb|tree| in our
  language which statically enforces the conventions.
\end{task}
\begin{sol}
\end{sol}

\section{Base Expression Language} \label{app:exp}
\subsection{Statics}
\subsection{Dynamics}
\end{document}
